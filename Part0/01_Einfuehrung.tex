\chapter[tocentry=Einführung, head=Einführung]{Einf\uech hrung}
\section[tocentry=Thematischer Überblick, head=Thematischer Überblick]{Thematischer \Uese berblick}
Die Magie -- oder präziser \emph{ars magica} -- ist eine sehr weitläufige und vielfältige Wissenschaft, der auf dem ersten Blick keine physikalisch sinnvolle Theorie zu Eigen ist. Der Prozess, eine vereinheitlichte Theorie der Magie zu entwickeln, ist bis heute noch nicht abgeschlossen, allerdings machen die theoretischen und analytischen Magier auf diesem Bereich große Fortschritte. Um dem Magus oder interessierten Laien diese Bemühungen nahezubringen, muss zuerst auf die Grundlagen eingegangen werden. \\
Die \emph{Empirice Magica} beginnt mit einer Entwicklung der nötigen Fachbegriffe und Ideen auf dem Gebiet der \emph{kosmologischen und sphärologischen Betrachtungen} der Welt, erhebt jedoch keinen Anspruch auf Vollständigkeit, lediglich die nötigsten Konzepte werden eingeführt.\\
Ausgerüstet mit diesem Wissen möge der Leser zur \emph{Magostatik} fortschreiten, wo die Idee des magischen Feldes und seiner Manipulationen entwickelt wird. Allerdings beschränken wir uns hier auf eine zeitunabhängige Theorie.
Die am häufigsten beobachtbaren magischen Phänomene (wie z. B. Zauber) sind jedoch zeitabhängig und können deshalb erst in der \emph{Magodynamik} eingeführt und untersucht werden.

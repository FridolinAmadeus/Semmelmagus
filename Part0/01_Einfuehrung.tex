\chapter[tocentry=Einführung, head=Einführung]{Einf\uech hrung}
\section[tocentry=Thematischer Überblick, head=Thematischer Überblick]{Thematischer \Uese berblick}
Die Magie -- oder präziser \emph{ars magica} -- ist eine sehr weitläufige und vielfältige Wissenschaft, der auf dem ersten Blick keine physikalisch sinnvolle Theorie zu Eigen ist. Der Prozess, eine vereinheitlichte Theorie der Magie zu entwickeln, ist bis heute noch nicht abgeschlossen, allerdings machen die theoretischen und analytischen Magier auf diesem Bereich große Fortschritte. Um dem Magus oder interessierten Laien diese Bemühungen nahezubringen, muss zuerst auf die Grundlagen eingegangen werden. \\
Die \emph{Empirice Magica} beginnt mit einer Entwicklung der nötigen Fachbegriffe und Ideen auf dem Gebiet der \emph{kosmologischen und sphärologischen Betrachtungen} der Welt, erhebt jedoch keinen Anspruch auf Vollständigkeit, lediglich die nötigsten Konzepte werden eingeführt.\\
Ausgerüstet mit diesem Wissen möge der Leser zur \emph{Magostatik} fortschreiten, wo die Idee des magischen Feldes und seiner Manipulationen entwickelt wird. Allerdings beschränken wir uns hier auf eine zeitunabhängige Theorie.
Die am häufigsten beobachtbaren magischen Phänomene (wie z. B. Zauber) sind jedoch zeitabhängig und können deshalb erst in der \emph{Magodynamik} eingeführt und untersucht werden.

\section[tocentry=Inhaltliche Einführung, head=Inhaltliche Einführung]{Inhaltliche Einf\uese hrung}
\begin{bla}{Bemerkung und Warnung}
Die hier angeführten Darstellungen sind (aufgrund der nichtquantitativen Untersuchung) grob vereinfachende Darstellungen des in dieser Abhandlung vorliegenden Inhalts und daher als vorläufig zu verstehen. Grundlage dieser Einführung sind die übersetzten und kommentierten \emph{Gespräche zwischen Rohal und dem General} \cite{rohalfragen}. 
\footnote{Die historische Einordnung und die Vertrauenswürdigkeit dieser Quelle sind allerdings zweifelhaft. \emph{Primum} bleibt \emph{der General} ohne Namen. Es liegt daher nahe, zu fragen, ob er überhaupt eine Person (oder eine literarische Figur) war. (In gewissen Kreisen vermutet man gar, der General sei der Name von Rohals Katze gewesen.) \emph{Secundum} ist unbekannt, ob (und wenn ja, warum) Rohal die Grundprinzipien der Magie seinem obersten General (oder einer anderen Person) erklärt hat. \emph{Tertium} weiß niemand, ob dieser Text tatsächlich aus Rohals Feder stammt (oder ob dies das Werk eines anderen Magus ist.) \emph{Quartum} weiß auch niemand, wer diesen Text übersetzt und kommentiert hat.}
\end{bla}

\begin{bla}{\textsc{General}: Was ist die Grundlage der \emph{ars magica}? \\
\textsc{Rohal}}
Die physikalische Grundlage des Wirkens von Magie beruht auf einem recht einfach anmutenden Erhaltungsprinzip, und zwar der \emph{Erhaltung von Sikaryan und Nayrakis}, welche in vielerlei Formulierungen relevant sein wird. Dabei stellt \emph{Sikaryan} das reine astrale Feld und \emph{Nayrakis} die im Kosmos vorliegende Ordnung dar. Der Prozess eines Zaubers drückt die Umwandlung von Sikaryan in Nayrakis aus.
\end{bla}
\begin{bla}{\textsc{General}: Und warum stellt dies jetzt einen Zauber dar? \\ 
\textsc{Rohal}}
Man nehme als Beispiel einen Flammenstrahl. Dieser wird dadurch erzeugt und aufrechterhalten, dass Prozesse in der Luft eintreten, die so unwahrscheinlich sind, dass sie nie eintreten würden, wenn man ihr nicht durch den Verbrauch von Sikaryan (also dem Zuführen von Astralenergie) aufrechterhalten würde. Durch das Erzwingen dieser Zauberwirkung wird jedoch (lokal) die Ordnung, also das Nayrakis erhöht. Dies geschieht in gleichem Maße, sodass die Summe von Sikaryan und Nayrakis konstant bleiben. 
\end{bla}
\begin{bla}{\textsc{General}: Was ist denn in diesem Beispiel aus der Energieerhaltung geworden? Zahlt der Zauberer diese Energie? \\ \textsc{Rohal}}
Nun, habt Geduld. Wenn Ihr mir gut zugehört habt, wüsstet Ihr, dass ich die Energie noch nicht erwähnt habe. Mit ihr verhält es sich so: Die Energie für den Zauber kommt nicht vom Zauberer selbst, daraus würden unerklärliche Paradoxien erwachsen \footnote{Zum Beispiel, wieso der Energieaufwand eines Zaubers im Allgemeinen nicht vom Abstaund zum Wirker abhängt: Dies implizierte, dass der Transport der Energie zum Zauberort selbst keine Energie kostet, was nicht plausibel erscheint.}, außerdem ist ein magisch erschöpfter Zauberer durchaus dazu in der Lage, profane Dinge zu tun, seine körperliche und seine Astralenergie sind (weitgehend) unabhängig voneinander. 
\end{bla}

\chapter[tocentry=Unwahrscheinlichkeitskalkül, head=Unwahrscheinlichkeitskalkül]{Unwahrscheinlichkeitskalk\uech l}
"`[Es ist] einer der gravierenden Fehler derer, die einer \emph{theoria magica} ablehnend gegenüberstehen, anzunehmen, das Eintreten eines magischen Ereignisses sei im Sinne einer Axiomatisierung der nichtmagischen Naturtheorie unmöglich. [...] Ich aber sage euch, dass [das Auftreten] eines [magischen] Ereignisses lediglich eine Frage der \emph{Unwahrscheinlichkeit}, nicht jedoch der \emph{Unmöglichkeit} ist. Eine [konsistente] theoria magica enthält die [nichtmagische Naturtheorie] als Limes der infinitesimalen\ldots [\textit{Rest unleserlich}]"' \cite{apokryphen}

Inwiefern \textsc{Rohal} bereits den Formalismus des Unwahrscheinlichkeitskalküls kannte, bleibt unklar. Er hat jedoch bereits die Begriffe der Unwahrscheinlichkeit und der Unwahrscheinlichkeitsamplitude definiert, also folgen wir hier \textsc{Rohal}.
\begin{bla}{Vorbemerkungen}
Im Folgenden sei stets ein beliebiger Wahrscheinlichkeitsraum $(\Omega,\mathcal{A},P)$ gegeben, d. h. ein Grundraum $\Omega$ aus Elementarereignissen mit einer $\sigma$-Algebra $\mathcal{A}$ über $\Omega$ und einem auf $\mathcal{A}$ definierten Wahrscheinlichkeitsmaß $P$. Als \emph{Ereignis} bezeichnen wir ein Element von $\mathcal{A}$.
\end{bla}
\begin{bla}{Definition: Unwahrscheinlichkeit $I$}
Unter der \emph{Unwahrscheinlichkeit} eines Ereignisses \textbf{E} versteht man den Kehrwert seiner Wahrscheinlichkeit:\\
\begin{equation}
I(\textbf{E}) := \frac{1}{P(\textbf{E})}
\end{equation}
$I$ nennt man \emph{Unwahrscheinlichkeitsmaß}\footnote{Genau genommen ist die Bezeichnung "`Unwahrscheinlichkeitsmaß"' für $I$ irreführend, da sie suggeriert, dass $I$ ein Maß ist, was nicht der Fall ist.} auf $\mathcal{A}$. Somit erhält man den \emph{Unwahrscheinlichkeitsraum} $(\Omega,\mathcal{A},I)$.
\end{bla}
\clearpage
\begin{bla}{Definition: Unwahrscheinlichkeitsamplitude $A$}
Die \emph{Unwahrscheinlichkeitsamplitude} $A$ eines Ereignisses \textbf{E} ist definiert als:
\begin{equation}
A(\textbf{E}) := \ln{I(\textbf{E})}
\end{equation}
\end{bla}
\begin{bla}{Bemerkungen}
Für die Wertebereiche der Größen gelten\footnote{Hierbei stehe \textbf{E} für ein tatsächlich mögliches Ereignis mit $P(\textbf{E}) > 0$, denn wie \textsc{Rohal} thematisieren wir hier die unwahrscheinlichen, nicht die unmöglichen Ereignisse.}:
\begin{itemize}
\item $P(\textbf{E}) \in ]0,1]$
\item $I(\textbf{E}) \in [1,\infty[$
\item $A(\textbf{E}) \in [0,\infty[$
\end{itemize}
Weiterhin gelten zwischen den Größen trivialerweise die folgenden Beziehungen:
\begin{itemize}
\item $A(\textbf{E}) = - \ln P(\textbf{E})$
\item $I(\textbf{E}) = \e ^ {A(\textbf{E})}$
\item $P(\textbf{E}) = I(\textbf{E}) ^{-1} = \e ^{-A(\textbf{E})}$
\end{itemize}
Die Unwahrscheinlichkeitsamplitude kann auch als Informationsgehalt eines Ereignisses interpretiert werden \cite{informationstheorie}.
\end{bla}
\begin{bsp}
Der Wahrscheinlichkeitsraum sei $([0,1],\mathcal{B}([0,1]),\lambda)$. Hierbei ist $\mathcal{B}([0,1])$ die \textsc{Borelius}-$\sigma$-Algebra über $[0,1]$ und $\lambda$ das \textsc{Lebexosch}-Maß auf $\mathcal{B}([0,1])$. Dies ist tatsächlich ein Wahrscheinlichkeitsmaß, da es $\sigma$-additiv ist und $\lambda([0,1])=1$. Das zugehörige Unwahrscheinlichkeitsmaß sei $I_\lambda$, sowie die Unwahrscheinlichkeitsamplitude $A_\lambda$. Dann ist beispielsweise für ein beliebiges, nichtleeres Intervall $[a,b]\su[0,1]$
\begin{align*}
I_\lambda([a,b])&=\frac{1}{\lambda([a,b])}=\frac{1}{b-a}\\
\intertext{und}
A_\lambda([a,b])&=\ln{I_\lambda([a,b])}=-\ln(b-a)\text{.}
\end{align*}
\end{bsp}

\begin{bla}{Definition: \textsc{Rohal}-Negentropie $R(\textbf{E})$}
Durch die Funktion $X: \Omega\to\R: \omega\mapsto A(\{\omega\})$ ist eine reellwertige Zufallsvariable gegeben. Sei \textbf{E} ein Ereignis.
\begin{equation}
R(\textbf{E}) := k_\text{R}*\int_\textbf{E} X\td I
\end{equation}
heißt \emph{\textsc{Rohal}-Negentropie} von \textbf{E}. Hierbei bezeichne $k_\text{R}$ die \textsc{Rohal}-Konstante.
\end{bla}
\chapter[tocentry=Astralfeldtheorie, head=Astralfeldtheorie]{Astralfeldtheorie 1}
"`Das \emph{Campus Physicus}! Dieses Modell wird unsere Vorstellung der \emph{vis magica} grundlegend verändern..."' (Autor unbekannt, wird aber Firlionel \textsc{Nachtschatten} zugeschrieben).

Die Astralfeldtheorie war einer der ersten Versuche eines konsistenten Modells der Meta- und analytischen Magie. Aus diesem Teilgebiet der \emph{magica theoretica} stammen bis heute Begriffe wie \emph{Kraftlinie} und \emph{Nodix}. Der eigentliche Erfolg dieser Theorie im Mikrokosmos begann durch die Vereinigung dieser Theorie mit der Quantenmechanik.  Ähnliche "`Feld"'modelle in der Kosmologie (s. unten) sollten zunächst nicht mit dieser Theorie verwechselt werden (auch wenn gewisse thematische Überschneidungen freilich kein Zufall sind...)
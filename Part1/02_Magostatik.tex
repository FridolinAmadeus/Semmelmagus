\chapter[tocentry=Magostatik, head=Magostatik]{Magostatik}
"`Das \emph{Campus Physicus}! Dieses Modell wird unsere Vorstellung der \emph{vis magica} grundlegend verändern..."' (Autor unbekannt, wird aber Firlionel \textsc{Nachtschatten} zugeschrieben).

Die Astralfeldtheorie war einer der ersten Versuche eines konsistenten Modells der Meta- und analytischen Magie. Aus diesem Teilgebiet der \emph{magica theoretica} stammen bis heute Begriffe wie \emph{Kraftlinie} und \emph{Nodix}. Der eigentliche Erfolg dieser Theorie im Mikrokosmos begann durch die Vereinigung dieser Theorie mit der Quantenmechanik.  Ähnliche "`Feld"'modelle in der Kosmologie (s. unten) sollten zunächst nicht mit dieser Theorie verwechselt werden (auch wenn gewisse thematische Überschneidungen freilich kein Zufall sind...)

\section{Motivation}
%Was ist Magie? Wie kann man sie beschreiben?
Eine der Kernideen der Magostatik und Magodynamik ist die Quantisierung der Astralenergie, welche durch sogenannte \emph{Sikaryonen} transportiert wird \cite{quanten}. Sie eignen sich vortrefflich, um die Wirkung von Magie zu erklären (näheres dazu in Metaphysica theoretica, Wechselwirkung der Sikaryonen). Aus ihrer Bewegung kann man das \emph{astrale Feld} oder \emph{Magiefeld} konstruieren.

\section[tocentry=Einführung in die Astralfeldtheorie, head=Einführung in die Astralfeldtheorie]{Einf\uese hrung in die Astralfeldtheorie}
\begin{bla}{Sikaryon und onus arcanum}
Das Sikaryon, benannt nach der Sphärenkraft \emph{Sikaryan}, durch welche die Astralkraft dargestellt wird, ist das Elementarteilchen der Magie. Es besitzt eine geringe Masse $m_\text{A}\approx10^{-29}$ kg\footnote{Die Abweichung beträgt maximal $1,2*10^{-27}$ kg, leider konnten noch keine genaueren Messungen durchgeführt werden.} und eine kleinste Einheit $O_\text{A}$ des \emph{onus arcanum} $O$ ($[O]=1$ $O_\text{A}$). Dies ist gewissermaßen das Vermögen, vom Astralfeld beeinflusst zu werden.
\end{bla}
\begin{bla}{Bemerkung}
Die einzigen bisher beobachteten und bekannten Objekte, die ein onus arcanum besitzen, sind Sikaryonen. Es konnte durch theoretische Überlegungen \cite{quanten} gezeigt werden, dass, falls es noch andere Teilchen mit onus arcanum gibt, deren onus arcanum ein Vielfaches von $O_\text{A}$ sein muss -- daher wurde dies als Einheit gewählt\footnote{Auf keinen Fall sollte, obwohl beides quantisiert ist, das onus arcanum mit astraler Energie gleichgesetzt werden. Letztere wird durch die Bewegung des onus arcanum im Astralfeld frei -- siehe \ref{astralenergie}.}.
\end{bla}
\begin{bla}{Das Astrale Feld}
Das astrale Feld oder Magiefeld $\vec{\As}$ ist ein alles durchdringendes Vektorfeld. Es kann beschrieben werden durch eine Abbildung
\begin{equation}
\vec{\As}: \K\to\Sp: \begin{pmatrix}\vec{r}\\\mathring{r}\\\gamma\\T\end{pmatrix}\mapsto\begin{pmatrix}\As_x\\\As_y\\\As_z\\\As_{\mathring{r}}\\\As_\gamma\end{pmatrix}\text{.}
\end{equation}
Die Einheit der astralen Feldstärke ist $[\As]=1$ Na (Nachtschatten, zu Ehren von Firlionel \textsc{Nachtschatten}s Vereinheitlichter Kräftetheorie \cite{nachtschatten}). 1 Na ist die Feldstärke, die auf ein Sikaryon eine Kraft von 1 N (\textsc{Newtosch}) ausübt.

Im astralen Feld wirkt analog zu anderen Feldern eine Kraft
\begin{equation}
\vec{F}_\text{Astral}=O\cdot\vec{\As}\label{eq:astralkraft}
\end{equation}
auf ein Objekt mit onus arcanum $O$.
\end{bla}
\begin{bla}{Bemerkung}
Ausgehend von Formel (\ref{eq:astralkraft}) kann die Einheit $O_A$ des onus arcanum dargestellt werden durch
\begin{equation}
O_\text{A}=\frac{1\text{ N}}{1\text{ Na}}\text{.}
\end{equation}
\end{bla}
\begin{bla}{Astrale Energie}\label{astralenergie}
Bewegt sich ein Objekt mit onus arcanum $O$ über einen Weg $\vec{s}$ in $\Sp$, so wird in Abhängigkeit des vorherrschenden astralen Feldes $\vec{\As}$ Energie aufgewandt oder frei. Diese ist gegeben durch:
\begin{equation}
E_\text{Astral}=\int_\text{Weg}\vec{F}\td\vec{s}=O*\int_\text{Weg}\vec{\As}\td\vec{s}
\end{equation}
Hierbei wird die Energie frei, falls das Vorzeichen positiv ist, andernfalls muss sie aufgewandt werden.
\end{bla}
\begin{bla}{Erzeugung des astralen Feldes - Teil 1}
Ein Objekt mit onus arcanum wird nicht nur vom astralen Feld beeinflusst, sondern erzeugt auch selbst ein Feld. Für ein auf einen Punkt konzentriertes onus arcanum $O$ im Ursprung von $\Sp$ ist die Feldstärke im Punkt $\vec{p}=(\vec{r},\mathring{r},\gamma)^\top$ gegeben durch
\begin{equation}
\vec{\As}=-\frac{1}{2\pi^2\alpha_0}*\frac{O}{(x^2+y^2+z^2+\mathring{r}^2+\gamma^2)^2}*\begin{pmatrix}\vec{r}\\\mathring{r}\\\gamma\end{pmatrix}=-\frac{O}{2\pi^2\alpha_0|\vec{p}|^3}*\hat{p}
\end{equation}
mit dem Einheitsvektor $\hat{p}=\frac{\vec{p}}{|\vec{p}|}$, falls der Punkt im Vakuum liegt. Hierbei ist $\alpha_0=6$,$33*10^{32}$ N\! m$^{-3}$\! Na$^{-2}$ die \emph{magische Feldkonstante}\footnote{Die Abweichung beträgt maximal $3,5*10^{30}$ N\! m$^{-3}$.}.
\end{bla}
\begin{bsp}
Ein Sikaryon erzeugt im Vakuum ein Feld, das im Abstand $d$ von einem Finger die Feldstärke
\begin{align*}
\As&=\frac{O}{2\pi^2\alpha_0d^3}=\frac{1\text{ }O_A}{2\pi^2*6{,}33*10^{32}\text{ N\! m}^{-3}\text{\! Na}^{-2}*(0{,}02\text{ m})^3}\approx10^{-29}\text{ Na}\\
\intertext{besitzt. Ein zweites Sikaryon in diesem Abstand zum ersten erfährt also ungefähr die Beschleunigung}
a_\text{Astral}&=\frac{F_\text{Astral}}{m}=\frac{1\text{ }O_A}{m_\text{A}}\As=\frac{1\text{ }O_A}{10^{-29}\text{ kg}}*10^{-29}\text{ Na}=1\text{ m\! s}^{-2}\text{.}
\end{align*}
\end{bsp}
\begin{bla}{Astraldichte}
Die \emph{Astraldichte} in einem fünfdimensionalen Hypervolumen $H$ in $\Sp$ ist definiert als der Quotient aus onus arcanum $O$, das sich in dem Hypervolumen befindet, und $H$:
\begin{equation}
\rho=\frac{O}{H},\ [\rho]=1\ O_A*1\text{ m}^{-5}
\end{equation}
Dies kann man für infitesimal kleine Hypervolumen verallgemeinern als Astraldichte in einem Punkt.\footnote{Dabei wird zwecks mathematischer Simplizität das vorliegende Diskontinuum ignoriert, selbstverständlich ist das onus arcanum immer noch gequantelt.}
\end{bla}
\begin{bla}{Erzeugung des Astralen Feldes - Teil 2}
Ist im Koordinatensystem $\Sp$ onus arcanum beliebig verteilt, ist das resultierende astrale Feld an einem Punkt $\vec{p}=(\vec{r},\mathring{r},\gamma)^\top$ im Vakuum gegeben durch
\begin{equation}
\vec{\As}=-\frac{1}{2\pi^2\alpha_0}\int_\Sp\frac{\rho(\vec{p}')}{|\vec{p}-\vec{p}'|^4}*(\vec{p}-\vec{p}')\td H,
\end{equation}
wobei $\vec{p}'$ stets die Position des Volumendifferentials $\td H$ darstellt.
\end{bla}
\section{Astraler Fluss und Induktion}
\begin{bla}{Magiepermittivität}
Die Stärke des astralen Feldes hängt vom Medium ab, in welchem das Feld betrachtet wird. Jedes Medium besitzt eine eigene \emph{relative Magiepermittivität} $\alpha_\text{r}$, welche die Schwächung (oder Stärkung) des Feldes darin misst. $\alpha_\text{r}$ ist dimensionslos ($[\alpha_\text{r}]=1$). Das Produkt $\alpha_0\alpha_r$ heißt \emph{(absolute) Magiepermittivität}. Das Feld im Medium wird gegenüber dem Feld im Vakuum um $\frac{1}{\alpha_\text{r}}$ abgeschwächt. Die Formel für die Berechnung beliebiger astraler Felder in beliebigen Medien ist somit:
\begin{equation}
\vec{\As}=-\frac{1}{2\pi^2\alpha_0\alpha_\text{r}}\int_\Sp\frac{\rho(\vec{p}')}{|\vec{p}-\vec{p}'|^4}*(\vec{p}-\vec{p}')\td V
\end{equation}
\end{bla}
\begin{bla}{Relative Permittivität verschiedener Medien}
Das Vakuum hat kanonisch eine relative Permittivität von 1. Luft und praktisch alle Gase haben einen Wert, der nur sehr knapp darüber liegt (in der Größenordnung $10^{-5}$), sodass man diese vernachlässigen kann. Bei anderen Medien kann $\alpha_r$ aber beträchtlich abweichen. Durch die sehr hohen Werte bei Metallen kommt so der "`Bann des Eisens"' genannte, magiedämpfende Effekt zustande. Im Limbus wird durch die Kompaktifizierung der drei Raum- und der Zeitdimension der Wert sehr gering, so dass das Feld dort viel stärker ist als anderswo. Einige experimentell bestimmte Werte sind in Tabelle \ref{tab:permitt} zu finden.
\end{bla}
\begin{table}[htb]
	\centering
		\begin{tabular}{c|c||c|c}
			Material&$\alpha_\text{r}$&Material&$\alpha_\text{r}$\\\hline\hline
			Vakuum&1&Eisen&124,56409\\\hline
			Luft&1,00002&Kupfer&174,02207\\\hline
			Sauerstoff&1,00001&Nickel&155,56999\\\hline
			Kohlenstoffdioxid&1,00004&Silber&98,03452\\\hline
			Stickstoff&1,00002&Gold&203,10495\\\hline
			Wasser&2,10544&Zwergensilber&1503,10293\\\hline
			feuchte Erde&5,44 bis 6,23&Zwergengold&10561,11702\\\hline
			Eis&1,99813&Mindorium&0,51142\\\hline
			Sand&4,13 bis 4,75&Arkanium&0,48901\\\hline
			Holz&6,78 bis 7,01&Endurium&0,24004\\\hline
			Stein&4,63 bis 10,12&Stahl&189,22 bis 613,78\\\hline
			Limbus&0,10351&Koschbasalt&4032,31005\\\hline
			Mondsilber&892,30335&Blei&78,30281
		\end{tabular}
	\caption{Relative Magiepermittivitäten $\alpha_\text{r}$ verschiedener Medien}
	\label{tab:permitt}
\end{table}
\begin{bla}{Astrale Flussdichte und Fluss}
Die \emph{astrale Flussdichte} $\vec{\Fd}$ ist definiert als das Produkt aus astraler Feldstärke und Magiepermittivität:
\begin{equation}
\vec{\Fd}:=\alpha_0\alpha_\text{r}\vec{\As}
\end{equation}
Sie beschreibt gewissermaßen die Dichte der Feldlinien bezüglich eines spatialen Supervolumens. Die Gesamtheit an Astralfeldlinien, die ein Volumen durchdringen, heißt \emph{astraler Fluss} $\Phi_\text{A}$. Der astrale Fluss durch ein solches Supervolumen $S$ in $\Sp$ ist
\begin{equation}
\Phi_\text{A}=\int_{S}\vec{\Fd}\td \vec{S},
\end{equation}
es tragen also am meisten die Anteile der Flussdichte, die in $\Sp$ normal auf das (orientierte) Volumen stehen, zum astralen Fluss bei.
\end{bla}
\begin{bla}{Bemerkung}
Betrachtet man den Fluss durch eine (geschlossene) Hyperoberfläche $\partial H$, kann man den Integralsatz von \textsc{Gaußosch} anwenden:
%Oberflächenintegral über Flussdichte=onus arcanum im Inneren
\begin{equation}
\Phi_\text{A}=\int_{\partial H}\vec{\Fd}\td\vec{S}=\int_H\langle\nabla,\vec{\Fd}\rangle\td H=\int_H\rho\td H=O
\end{equation}
Dabei ist $O$ die Gesamtheit des im Inneren des Hypervolumens $H$ befindliche onus arcanum.
\end{bla}
\begin{bla}{Zauberhaftigkeit}
%Flussänderung als Indikator für Zauberei
\end{bla}
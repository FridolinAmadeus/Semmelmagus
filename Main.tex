\documentclass[11pt,a4paper,headings=optiontohead]{scrbook}
\usepackage[utf8]{inputenc}
\usepackage{tgschola}
\usepackage{tgothic}
\usepackage[T1]{fontenc}
\usepackage[ngerman]{babel}
%\usepackage[a4paper, total={5.5in, 8in}]{geometry}
\usepackage[onehalfspacing]{setspace}
\usepackage{anyfontsize}
\usepackage{tocloft}

% -- Mathe
%\usepackage{gensymb}
%fancyhdr, lastpage, booktabs, xy
\usepackage{graphicx}
\usepackage{amsmath}
\usepackage{amsfonts}
\usepackage{amssymb}
\usepackage{amsthm}
\usepackage{rotating}
\usepackage{hyperref}
\usepackage{mathtools}
\usepackage{marginnote}
\usepackage{MnSymbol}

% -- Malen
\usepackage{tikz}
\usetikzlibrary{patterns, decorations.pathreplacing, arrows}
\usepackage{units}

%\renewcommand{\familydefault}{\rmdefault}

%----Inhaltsverzeichnis- und Überschriftenformatierung----
\setkomafont{disposition}{\tgothfamily}
\renewcommand{\cfttoctitlefont}{\rmfamily\Large\bfseries}
\renewcommand{\cftpartfont}{\rmfamily\bfseries}
\renewcommand{\cftpartpagefont}{\rmfamily\bfseries}
\renewcommand{\cftchapfont}{\rmfamily}
\renewcommand{\cftchappagefont}{\rmfamily}

\makeatletter
\newcommand{\xeq}[2][]{\ensuremath{\overset{\mathclap
{\scriptscriptstyle{#2}}}{\underset{\mathclap{#1}}{=}}}}
\newcommand{\xleq}[2][]{\ensuremath{\overset{\mathclap
{\scriptscriptstyle{#2}}}{\underset{\mathclap{#1}}{\leq}}}}
\newcommand{\leqnomode}{\tagsleft@true\let\veqno\@@leqno}
\newcommand{\reqnomode}{\tagsleft@false\let\veqno\@@eqno}
\makeatother

\newcommand{\thistheoremname}{}

\newcommand{\C}{\mathbb{C}}
\newcommand{\R}{\mathbb{R}}
\newcommand{\N}{\mathbb{N}}
\newcommand{\Z}{\mathbb{Z}}
\newcommand{\K}{\mathbb{K}}
\newcommand{\Sp}{\mathbb{S}}

\newcommand{\A}{\:\:\forall}
\newcommand{\E}{\:\:\exists}

\newcommand{\su}{\subseteq}
\newcommand{\us}{\supseteq}
\newcommand{\ot}{\leftarrow}
\newcommand{\To}{\Rightarrow}
\newcommand{\Ot}{\Leftarrow}
\newcommand{\Oto}{\Leftrightarrow}

\renewcommand{\Re}{\operatorname{Re}}
\renewcommand{\Im}{\operatorname{Im}}

\newcommand{\Bild}{\operatorname{Bild}}
\newcommand{\Res}{\operatorname{Res}}

\newcommand{\e}{\text{\rmfamily e}}
\renewcommand{\i}{\text{\rmfamily i}}

\newcommand{\egn}{\varepsilon > 0}
\newcommand{\de}{\delta_\varepsilon}
\newcommand{\Ne}{\N_\varepsilon}

\newcommand{\tdb}{differenzierbar}
\newcommand{\tdbl}{differenzierbar }
\newcommand{\td}[1]{\text{\rmfamily d} #1}

\newcommand{\diff}[3][]{\frac{\text{\rmfamily d}^{#1} #2}{\text{\rmfamily d} #3^{#1}}}
\newcommand{\piff}[3][]{\frac{\partial^{#1} #2}{\partial #3 ^{#1}}}
\newcommand{\fehler}[3][]{\left| \piff[#1]{#2}{#3} \right| * \Delta #3}

%ä, ö, ü in Kapitelüberschriften
\newcommand{\aech}{a\kern-0.17em{e}}
\newcommand{\oech}{o\kern-0.13em{e}}
\newcommand{\uech}{u\kern-0.17em{e}}
\newcommand{\Aech}{A\kern-0.23em{e}}
%ä, ö, ü in Abschnittsüberschriften
\newcommand{\aese}{a\kern-0.22em{e}}
\newcommand{\oese}{o\kern-0.18em{e}}
\newcommand{\uese}{u\kern-0.22em{e}}
\newcommand{\Aese}{A\kern-0.26em{e}}
\newcommand{\Uese}{U\kern-0.19em{e}}

\newcommand{\As}{\mathfrak{A}}
%\newcommand{\As}{\text{\tgothfamily A}}
\newcommand{\Fd}{\mathfrak{F}}
%\newcommand{\Fd}{\text{\tgothfamily F}}

%`Text` im Mathemodus
\mathcode`\`="8000
\begingroup
\uccode`\~`\`
\uppercase{%
\endgroup
\def~#1`{\ \text{#1} \ }}
%Malpunkte
\mathcode`\*="8000
{\catcode`\*\active\gdef*{\cdot}}
%Punkte zu Kommata
\DeclareMathSymbol{.}{\mathord}{letters}{"3B}

\newtheoremstyle{paris}
  {\topsep}% Platzhalter nach oben
  {\topsep}% Platzhalter nach unten
  {\normalfont}% Schriftart im Theoremkörper
  {-25pt}% Seitlicher Einschub
  {\bfseries}% Schriftart in der Theoremüberschrift
  {:\\}% Punkt zwischen Überschrift und Körper
  {5pt plus 1pt minus 1pt}% Platz nach überschrift mit 2pt Ausweichvermögen
  {}% Spezifikationen

\swapnumbers
\theoremstyle{paris}
\newtheorem{df}{Definition}[chapter]
\newtheorem{sz}[df]{Satz}
\newtheorem{genericthm}[df]{\thistheoremname}
\newtheorem{bsp}[df]{Beispiel}
\newenvironment{bw}[1][\textbf{Beweis}]{\begin{proof}[#1]}{\end{proof}}

%Bedienung: \begin{bla}{Satz von Horst}
\newenvironment{bla}[1]
{\renewcommand{\thistheoremname}{#1}\begin{genericthm}}{\end{genericthm}}

%----TEST-Kram----

\def\mathnote#1{%
  \tag*{\rlap{\hspace\marginparsep\smash{\parbox[t]{\marginparwidth}{%
  \footnotesize#1}}}}
}
\def\lmathnote#1{%
  \leqnomode%
  \tag*{\llap{\hspace\marginparsep\smash{\parbox[t]{\marginparwidth}{%
  \footnotesize#1}}}}%
  \reqnomode
}

\setlength\emergencystretch{1em}

%----Dokument----

\begin{document}
\begin{titlepage}
\centering
\phantom{Semmelmagus}
\vspace{5cm}
{\tgothfamily\fontsize{60}{72}\selectfont De theoria arcana\par}
\vspace{1cm}
{\Large Paul \textsc{Schwahn}, Jann \textsc{van der Meer}\par}
\vfill
\end{titlepage}
\tableofcontents
\clearpage
\setcounter{chapter}{-1}
\chapter[tocentry=Einführung, head=Einführung]{Einf\uech hrung}
\section[tocentry=Thematischer Überblick, head=Thematischer Überblick]{Thematischer \Uese berblick}
Die Magie -- oder präziser \emph{ars magica} -- ist eine sehr weitläufige und vielfältige Wissenschaft, der auf dem ersten Blick keine physikalisch sinnvolle Theorie zu Eigen ist. Der Prozess, eine vereinheitlichte Theorie der Magie zu entwickeln, ist bis heute noch nicht abgeschlossen, allerdings machen die theoretischen und analytischen Magier auf diesem Bereich große Fortschritte. Um dem Magus oder interessierten Laien diese Bemühungen nahezubringen, muss zuerst auf die Grundlagen eingegangen werden. \\
Die \emph{Empirice Magica} beginnt mit einer Entwicklung der nötigen Fachbegriffe und Ideen auf dem Gebiet der \emph{kosmologischen und sphärologischen Betrachtungen} der Welt, erhebt jedoch keinen Anspruch auf Vollständigkeit, lediglich die nötigsten Konzepte werden eingeführt.\\
Ausgerüstet mit diesem Wissen möge der Leser zur \emph{Magostatik} fortschreiten, wo die Idee des magischen Feldes und seiner Manipulationen entwickelt wird. Allerdings beschränken wir uns hier auf eine zeitunabhängige Theorie.
Die am häufigsten beobachtbaren magischen Phänomene (wie z. B. Zauber) sind jedoch zeitabhängig und können deshalb erst in der \emph{Magodynamik} eingeführt und untersucht werden.
\clearpage
\part{Empirice magica}
\chapter[tocentry=Sphärologische Betrachtungen, head=Sphärologische Betrachtungen]{Sph\aech rologische Betrachtungen}
%Quellen der Magie...
\section{Mathematische Betrachtung der Dimensionen}
Die sieben Dimensionen - drei Raumachsen sowie je eine Zeit-, Kausal-, Sphären- und Globulenachse - sind hinlänglich bekannt. Diese sollen im Folgenden formalisiert werden.
\begin{bla}{Raumdimensionen zum Ersten}
Der dreidimensionale Raum, in dem sich ein Normalsterblicher bewegt, ähnelt zwar eher einer 3-Sphäre, ist aber lokal homöomorph zum \textsc{Euklidius}schen Vektorraum $\R^3$ und kann somit lokal als dreiachsiges Koordinatensystem dargestellt werden. Im Folgenden beschreiben wir einen Punkt im Raum mit dem Vektor $\vec{r}=(x,y,z)^\top$, wobei $x$, $y$ und $z$ die Koordinaten des Punktes seien.
\end{bla}
\begin{bla}{Zeitliche Dimensionen}
Die Dimension der gewöhnlichen Zeit wird fortan mit der Koordinate $t\in\R$ beschrieben. Da die nodokausale Richtung nicht nur orthogonal auf die Zeit steht, sondern auch direkt mit ihr zusammenhängt, bietet es sich an, Zeitachse und Kausalitätsachse zu einer Ebene zusammenzufassen. Daraus resultiert die \emph{komplexe Zeit} $T$, die sich schreiben lässt als $T=t+\i\tau$. Somit wird die Koordinate $\tau\in\R$ der Kausalitätsachse als imaginäre Zeit aufgefasst. Dieses Modell mag zunächst seltsam erscheinen, jedoch ergeben sich dadurch praktische Vorteile bei der weiteren mathematischen Behandlung.
\end{bla}
\begin{bla}{Raumdimensionen zum Zweiten}
Auch die orthosphärische und transglobule Dimension können als Raumrichtungen aufgefasst werden, welche von gewöhnlichen Lebewesen i. A. nicht wahrzunehmen sind. Die Koordinate $\mathring{r}$ der Sphärenachse ist äquivalent zu einem Radius, der die Entfernung des jeweiligen Punktes zum Weltenherz beschreibt. Da letzteres unzugänglich, unantastbar und ohne Ausdehnung ist, liegt es nahe, den Halbraum $\R^+=\{\mathring{r}\in\R|\mathring{r}>0\}$ als Sphärenachse einzuführen. Keine Limitation erfährt jedoch die Koordinate $\gamma$ der Globulenachse, welche man also mit $\R$ beschreibt.
\end{bla}
\begin{bla}{Zusammenfassung des Raums}
Lässt man die zeitlichen Dimensionen außen vor und betrachtet nur räumliche Dimensionen, so erhält man das kartesische Produkt
\begin{equation}
\Sp:=\R^3\times\R^+\times\R
\end{equation}
(von bosparanisch \emph{spatium} = Raum), welches \emph{spatialer Momentankosmos} heißt.
\end{bla}
\begin{bla}{Der Kosmos}
Betrachtet man alle sieben Dimensionen zusammen, erhält man das kartesische Produkt
\begin{equation}
\K:=\R^3\times\R^+\times\R\times\C=\Sp\times\C
\end{equation}
(von aurelianisch \emph{kósmos} = Universum), den \emph{Kosmos}. Er beschreibt alles, was zu jeder Zeit möglich ist.
\end{bla}
\begin{bla}{Begriffe}
Im Folgenden sollen folgende Begriffe einheitlich verwendet werden:
\begin{itemize}
\item
Weg $\vec{s}=(s_1,s_2,s_3,s_4,s_5)^\top$ in $\Sp$, gegeben als Abbildung $\vec{s}: [0,1]\to\Sp$ (oder in Bogenlängenparametrisierung $[0,L]\to\Sp$)
\item
Zweidimensionale Fläche $A$ (oder -- eingebettet in mehr Dimensionen -- mit Orientierung: $\vec{A}$)
\item
Eindimensionaler Rand $\partial A$ einer Fläche $A$ (auch als Weg zu betrachten)
\item
Dreidimensionales Volumen $V$ (oder -- eingebettet in mehr Dimensionen -- mit Orientierung: $\vec{V}$)
\item
Zweidimensionale Oberfläche $\partial V$ eines Volumens $V$
\item
Vierdimensionales Supervolumen $S$ (oder -- eingebettet in mehr Dimensionen -- mit Orientierung $\vec{S}$)
\item
Dreidimensionale Superoberfläche $\partial S$ eines Supervolumens $S$
\item
Fünfdimensionales Hypervolumen $H$
\item
Vierdimensionale Hyperoberfläche $\partial H$ eines Hypervolumens $H$
\end{itemize}
Die Orientierung von Flächen, Volumen und Supervolumen im fünfdimensionalen Raum ist, z. B. bei einer gekrümmten Fläche, nicht zwangsläufig gegeben. Differentielle Flächen-, Volumen- und Supervolumenelemente $\td\vec{A}$, $\td\vec{V}$ und $\td\vec{S}$ besitzen jedoch immer eine Orientierung.
\end{bla}\clearpage
\chapter[tocentry=Magostatik, head=Magostatik]{Magostatik}
"`Das \emph{Campus Physicus}! Dieses Modell wird unsere Vorstellung der \emph{vis magica} grundlegend verändern..."' (Autor unbekannt, wird aber Firlionel \textsc{Nachtschatten} zugeschrieben).

Die Astralfeldtheorie war einer der ersten Versuche eines konsistenten Modells der Meta- und analytischen Magie. Aus diesem Teilgebiet der \emph{magica theoretica} stammen bis heute Begriffe wie \emph{Kraftlinie} und \emph{Nodix}. Der eigentliche Erfolg dieser Theorie im Mikrokosmos begann durch die Vereinigung dieser Theorie mit der Quantenmechanik.  Ähnliche "`Feld"'modelle in der Kosmologie (s. unten) sollten zunächst nicht mit dieser Theorie verwechselt werden (auch wenn gewisse thematische Überschneidungen freilich kein Zufall sind...)

\section{Motivation}
%Was ist Magie? Wie kann man sie beschreiben?
Eine der Kernideen der Magostatik und Magodynamik ist die Quantisierung der Astralenergie, welche durch sogenannte \emph{Astralonen} transportiert wird \cite{quanten}. Sie eignen sich vortrefflich, um die Wirkung von Magie zu erklären (näheres dazu in Metaphysica theoretica, Wechselwirkung der Astralonen). Aus ihrer Bewegung kann man das \emph{Astrale Feld} oder \emph{Magiefeld} konstruieren.

\section[tocentry=Einführung in die Astralfeldtheorie, head=Einführung in die Astralfeldtheorie]{Einf\uese hrung in die Astralfeldtheorie}
\begin{bla}{Astralon und onus arcanum}
Das Astralon ist das Elementarteilchen der Magie. Es besitzt eine geringe Masse $m_\text{A}\approx10^{-29}$ kg\footnote{Die Abweichung beträgt maximal $1,2*10^{-27}$ kg, leider konnten noch keine genaueren Messungen durchgeführt werden.} und eine kleinste Einheit $O_\text{A}$ des \emph{onus arcanum} $O$ ($[O]=1$ $O_\text{A}$). Dies ist gewissermaßen das Vermögen, vom Astralfeld beeinflusst zu werden.
\end{bla}
\begin{bla}{Bemerkung}
Die einzigen bisher beobachteten und bekannten Objekte, die ein onus arcanum besitzen, sind Astralonen. Es konnte durch theoretische Überlegungen \cite{quanten} gezeigt werden, dass, falls es noch andere Teilchen mit onus arcanum gibt, deren onus arcanum ein Vielfaches von $O_\text{A}$ sein muss -- daher wurde dies als Einheit gewählt\footnote{Auf keinen Fall sollte, obwohl beides quantisiert ist, das onus arcanum mit astraler Energie gleichgesetzt werden. Letztere wird durch die Bewegung des onus arcanum im Astralfeld frei -- siehe \ref{astralenergie}.}.
\end{bla}
\begin{bla}{Das Astrale Feld}
Das Astrale Feld oder Magiefeld $\vec{\As}$ ist ein alles durchdringendes Vektorfeld. Es kann beschrieben werden durch eine Abbildung
\begin{equation}
\vec{\As}: \K\to\Sp: \begin{pmatrix}\vec{r}\\\mathring{r}\\\gamma\\T\end{pmatrix}\mapsto\begin{pmatrix}\As_x\\\As_y\\\As_z\\\As_{\mathring{r}}\\\As_\gamma\end{pmatrix}\text{.}
\end{equation}
Die Einheit der astralen Feldstärke ist $[\As]=1$ Na (Nachtschatten, zu Ehren von Firlionel \textsc{Nachschatten}s Vereinheitlichter Kräftetheorie \cite{nachtschatten}). 1 Na ist die Feldstärke, die auf ein Astralon eine Kraft von 1 N (\textsc{Newtosch}) ausübt.

Im Astralen Feld wirkt analog zu anderen Feldern eine Kraft
\begin{equation}
\vec{F}_\text{Astral}=O\cdot\vec{\As}\label{eq:astralkraft}
\end{equation}
auf ein Objekt mit onus arcanum $O$.
\end{bla}
\begin{bla}{Bemerkung}
Ausgehend von Formel (\ref{eq:astralkraft}) kann die Einheit $O_A$ des onus arcanum dargestellt werden durch
\begin{equation}
O_\text{A}=\frac{1\text{ N}}{1\text{ Na}}\text{.}
\end{equation}
\end{bla}
\begin{bla}{Astrale Energie}\label{astralenergie}
Bewegt sich ein Objekt mit onus arcanum $O$ über einen Weg $\vec{s}$ in $\Sp$, so wird in Abhängigkeit des vorherrschenden Astralen Feldes $\vec{\As}$ Energie aufgewandt oder frei. Diese ist gegeben durch:
\begin{equation}
E_\text{Astral}=\int_\text{Weg}\vec{F}\td\vec{s}=O*\int_\text{Weg}\vec{\As}\td\vec{s}
\end{equation}
Hierbei wird die Energie frei, falls das Vorzeichen positiv ist, andernfalls muss sie aufgewandt werden.
\end{bla}
\begin{bla}{Erzeugung des Astralen Feldes - Teil 1}
Ein Objekt mit onus arcanum wird nicht nur vom Astralen Feld beeinflusst, sondern erzeugt auch selbst ein Feld. Für ein auf einen Punkt konzentriertes onus arcanum $O$ im Ursprung von $\Sp$ ist die Feldstärke im Punkt $\vec{p}=(\vec{r},\mathring{r},\gamma)^\top$ gegeben durch
\begin{equation}
\vec{\As}=-\frac{1}{2\pi^2\alpha_0}*\frac{O}{(x^2+y^2+z^2+\mathring{r}^2+\gamma^2)^2}*\begin{pmatrix}\vec{r}\\\mathring{r}\\\gamma\end{pmatrix}=-\frac{O}{2\pi^2\alpha_0|\vec{p}|^3}*\hat{p}
\end{equation}
mit dem Einheitsvektor $\hat{p}=\frac{\vec{p}}{|\vec{p}|}$. Hierbei ist $\alpha_0=6$,$33*10^{32}$ N\! m$^{-3}$\! Na$^{-2}$ die \emph{magische Feldkonstante}\footnote{Die Abweichung beträgt maximal $3,5*10^{30}$ N\! m$^{-3}$.}.
\end{bla}
\begin{bsp}
Ein Astralon erzeugt ein Feld, das im Abstand $d$ von einem Finger die Feldstärke
\begin{align*}
\As&=\frac{O}{2\pi^2\alpha_0d^3}=\frac{1\text{ }O_A}{2\pi^2*6{,}33*10^{32}\text{ N\! m}^{-3}\text{\! Na}^{-2}*(0{,}02\text{ m})^3}\approx10^{-29}\text{ Na}\\
\intertext{besitzt. Ein zweites Astralon in diesem Abstand zum ersten erfährt also ungefähr die Beschleunigung}
a_\text{Astral}&=\frac{F_\text{Astral}}{m}=\frac{1\text{ }O_A}{m_\text{A}}\As=\frac{1\text{ }O_A}{10^{-29}\text{ kg}}*10^{-29}\text{ Na}=1\text{ m\! s}^{-2}\text{.}
\end{align*}
\end{bsp}
\begin{bla}{Astraldichte}
Die \emph{Astraldichte} in einem Hypervolumen $V$ in $\Sp$ ist definiert als der Quotient aus onus arcanum $O$, das sich in dem Hypervolumen befindet, und $V$:
\begin{equation}
\rho=\frac{O}{V},\ [\rho]=1\ O_A*1\text{ m}^{-5}
\end{equation}
Dies kann man für infitesimal kleine Hypervolumen verallgemeinern als Astraldichte in einem Punkt.\footnote{Dabei wird zwecks mathematischer Simplizität das vorliegende Diskontinuum ignoriert, selbstverständlich ist das onus arcanum immer noch gequantelt.}
\end{bla}
\begin{bla}{Erzeugung des Astralen Feldes - Teil 2}
Ist im Koordinatensystem $\Sp$ onus arcanum beliebig verteilt, ist das resultierende arkane Feld an einem Punkt $\vec{p}=(\vec{r},\mathring{r},\gamma)^\top$ gegeben durch
\begin{equation}
\vec{\As}=-\frac{1}{2\pi^2\alpha_0}\int_\Sp\frac{\rho(\vec{p}')}{|\vec{p}-\vec{p}'|^4}*(\vec{p}-\vec{p}')\td V,
\end{equation}
wobei $\vec{p}'$ stets die Position des Volumendifferentials $\td V$ darstellt.
\end{bla}
\section{Astraler Fluss und Induktion}
\begin{bla}{Magiepermeabilität}

\end{bla}
\begin{bla}{Bemerkung}
%Bann des Eisens, hohe Permeabilität im Limbus
\end{bla}
\begin{bla}{Astrale Flussdichte und Fluss}

\end{bla}
\begin{bla}{Zauberhaftigkeit}
%Flussänderung als Indikator für Zauberei
\end{bla}\clearpage
\part{Viae mathematicae artis magicae}
\chapter[tocentry=Unwahrscheinlichkeitskalkül, head=Unwahrscheinlichkeitskalkül]{Unwahrscheinlichkeitskalk\uech l}
"`[Es ist] einer der gravierenden Fehler derer, die einer theoria magica ablehnend gegenüberstehen, anzunehmen, das Eintreten eines magischen Ereignisses sei im Sinne einer Axiomatisierung der nichtmagischen Naturtheorie unmöglich. [...] Ich aber sage euch, dass [das Auftreten] eines [magischen] Ereignisses lediglich eine Frage der \emph{Unwahrscheinlichkeit}, nicht jedoch der \emph{Unmöglichkeit} ist. Eine [konsistente] theoria magica enthält die [nichtmagische Naturtheorie] als Limes der infinitesimalen\ldots [\textit{Rest unleserlich}]"' (Zit. aus "`\emph{Rohals Apokryphen}"', Universitätsbibliothek zu Punin)

Inwiefern \textsc{Rohal} bereits den Formalismus des Unwahrscheinlichkeitskalküls kannte, bleibt unklar. Er hat jedoch bereits die Begriffe der Unwahrscheinlichkeit und der Unwahrscheinlichkeitsamplitude definiert, also folgen wir hier \textsc{Rohal}.
\begin{bla}{Vorbemerkungen}
Im Folgenden sei stets ein beliebiger Wahrscheinlichkeitsraum $(\Omega,\mathcal{A},P)$ gegeben, d. h. ein Grundraum $\Omega$ aus Elementarereignissen mit einer $\sigma$-Algebra $\mathcal{A}$ über $\Omega$ und einem auf $\mathcal{A}$ definierten Wahrscheinlichkeitsmaß $P$. Als \emph{Ereignis} bezeichnen wir ein Element von $\mathcal{A}$.
\end{bla}
\begin{bla}{Definition: Unwahrscheinlichkeit $I$}
Unter der \emph{Unwahrscheinlichkeit} eines Ereignisses \textbf{E} versteht man den Kehrwert seiner Wahrscheinlichkeit:\\
\begin{equation}
I(\textbf{E}) := \frac{1}{P(\textbf{E})}
\end{equation}
$I$ nennt man \emph{Unwahrscheinlichkeitsmaß}\footnote{Im Grunde genommen ist die Bezeichnung "`Unwahrscheinlichkeitsmaß"' für $I$ irreführend, da sie suggeriert, dass $I$ ein Maß ist, was nicht der Fall ist.} auf $\mathcal{A}$. Somit erhält man den \emph{Unwahrscheinlichkeitsraum} $(\Omega,\mathcal{A},I)$.
\end{bla}
\clearpage
\begin{bla}{Definition: Unwahrscheinlichkeitsamplitude $A$}
Die \emph{Unwahrscheinlichkeitsamplitude} $A$ eines Ereignisses \textbf{E} ist definiert als:
\begin{equation}
A(\textbf{E}) := \ln{I(\textbf{E})}
\end{equation}
\end{bla}
\begin{bla}{Bemerkungen}
Für die Wertebereiche der Größen gelten\footnote{Hierbei stehe \textbf{E} für ein tatsächlich mögliches Ereignis mit $P(\textbf{E}) > 0$, denn wie \textsc{Rohal} thematisieren wir hier die unwahrscheinlichen, nicht die unmöglichen Ereignisse.}:
\begin{itemize}
\item $P(\textbf{E}) \in ]0,1]$
\item $I(\textbf{E}) \in [1,\infty[$
\item $A(\textbf{E}) \in [0,\infty[$
\end{itemize}
Weiterhin gelten zwischen den Größen trivialerweise die folgenden Beziehungen:
\begin{itemize}
\item $I(\textbf{E}) = - \ln P(\textbf{E})$
\item $I(\textbf{E}) = \e ^ {A(\textbf{E})}$
\item $P(\textbf{E}) = I(\textbf{E}) ^{-1} = \e ^{-A(\textbf{E})}$
\end{itemize}
\end{bla}
\begin{bsp}
Der Wahrscheinlichkeitsraum sei $([0,1],\mathcal{B}([0,1]),\lambda)$. Hierbei ist $\mathcal{B}([0,1])$ die \textsc{Borelius}-$\sigma$-Algebra über $[0,1]$ und $\lambda$ das \textsc{Lebexosch}-Maß auf $\mathcal{B}([0,1])$. Dies ist tatsächlich ein Wahrscheinlichkeitsmaß, da es $\sigma$-additiv ist und $\lambda([0,1])=1$. Das zugehörige Unwahrscheinlichkeitsmaß sei $I_\lambda$, sowie die Unwahrscheinlichkeitsamplitude $A_\lambda$. Dann ist beispielsweise für ein beliebiges Intervall $[a,b]\in[0,1]$
\begin{align*}
I_\lambda([a,b])&=\frac{1}{\lambda([a,b])}=\frac{1}{b-a}\\
\intertext{und}
A_\lambda([a,b])&=\ln{I_\lambda([a,b])}=-\ln(b-a)\text{.}
\end{align*}
\end{bsp}
\begin{bla}{Definition: \textsc{Rohal}-Negentropie $R(\textbf{E})$}
Sei \textbf{E} ein Ereignis.
\begin{equation}
R(\textbf{E}) := k_\text{R} * A(\textbf{E})
\end{equation}
heißt \emph{\textsc{Rohal}-Negentropie} von \textbf{E}. Hierbei bezeichne $k_\text{R}$ die \textsc{Rohal}-Konstante.
\end{bla}\clearpage
\part{Metaphysica theoretica}
\chapter{Mathematische Probleme der Magietheorie}
\textsc{Leskatosch}: "`Mit gesundem Zwergenverstand kommt man ziemlich weit"'. \cite{leskatosch}\clearpage
\part{Implicationes arcanobiologicae}

%\input{}\clearpage

\begin{thebibliography}{9}
\addcontentsline{toc}{part}{Literaturverzeichnis}
\bibitem{quanten} Marcus \textsc{Hamilton}, \emph{De vis astralis}, Akademie der Hohen Magie zu Punin, 1037 BF
\bibitem{erzzwerge} \textsc{Leskolombrosch}, Sohn des \textsc{Leskándin} (Hrsg.), \emph{Gesammelte Mathematik der Erzzwerge}, Königliche Bibliothek von Xorlosch, 825 BF
\bibitem{nachtschatten} Firlionel \textsc{Nachtschatten}, \emph{Vereinheitlichte Kräftetheorie}, Akademie der Geistigen Kraft zu Fasar
\bibitem{apokryphen} Apokryphen von \textsc{Rohal} dem Weisen, Akademie der Hohen Magie zu Punin, um 580 BF
\bibitem{informationstheorie} \textsc{Turgrim}, Sohn des \textsc{Tietzax}, \emph{Die Theorie der Information}, Königliche Bibliothek von Xorlosch, 763 BF
\bibitem{leskatosch} \textsc{Leskatosch}, Sohn des \textsc{Leskolombrosch}, \emph{Fachdidaktische Abhandlung zur erzzwergischen Mathematik}, Königliche Bibliothek von Xorlosch, 914 BF
\bibitem{rohalfragen} Unbekannt, \emph{Was ist die Grundlage der \emph{ars magica}? - Die Gespräche zwischen Rohal und dem General}, Akademie der Hohen Magie zu Punin, um 580 BF
\end{thebibliography}
\end{document}

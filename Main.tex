\documentclass[11pt,a4paper,headings=optiontohead]{scrbook}
\usepackage[utf8]{inputenc}
\usepackage{tgschola}
\usepackage{tgothic}
\usepackage[T1]{fontenc}
\usepackage[ngerman]{babel}
%\usepackage[a4paper, total={5.5in, 8in}]{geometry}
\usepackage[onehalfspacing]{setspace}
\usepackage{anyfontsize}
\usepackage{tocloft}

% -- Mathe
%\usepackage{gensymb}
%fancyhdr, lastpage, booktabs, xy
\usepackage{graphicx}
\usepackage{amsmath}
\usepackage{amsfonts}
\usepackage{amssymb}
\usepackage{amsthm}
\usepackage{rotating}
\usepackage{hyperref}
\usepackage{mathtools}
\usepackage{marginnote}
\usepackage{MnSymbol}

% -- Malen
\usepackage{tikz}
\usetikzlibrary{patterns, decorations.pathreplacing, arrows}
\usepackage{units}

%\renewcommand{\familydefault}{\rmdefault}

%----Inhaltsverzeichnis- und Überschriftenformatierung----
\setkomafont{disposition}{\tgothfamily}
\renewcommand{\cfttoctitlefont}{\rmfamily\Large\bfseries}
\renewcommand{\cftpartfont}{\rmfamily\bfseries}
\renewcommand{\cftpartpagefont}{\rmfamily\bfseries}
\renewcommand{\cftchapfont}{\rmfamily}
\renewcommand{\cftchappagefont}{\rmfamily}

\makeatletter
\newcommand{\xeq}[2][]{\ensuremath{\overset{\mathclap
{\scriptscriptstyle{#2}}}{\underset{\mathclap{#1}}{=}}}}
\newcommand{\xleq}[2][]{\ensuremath{\overset{\mathclap
{\scriptscriptstyle{#2}}}{\underset{\mathclap{#1}}{\leq}}}}
\newcommand{\leqnomode}{\tagsleft@true\let\veqno\@@leqno}
\newcommand{\reqnomode}{\tagsleft@false\let\veqno\@@eqno}
\makeatother

\newcommand{\thistheoremname}{}

\newcommand{\C}{\mathbb{C}}
\newcommand{\R}{\mathbb{R}}
\newcommand{\N}{\mathbb{N}}
\newcommand{\Z}{\mathbb{Z}}

\newcommand{\A}{\:\:\forall}
\newcommand{\E}{\:\:\exists}

\newcommand{\su}{\subseteq}
\newcommand{\us}{\supseteq}
\newcommand{\ot}{\leftarrow}
\newcommand{\To}{\Rightarrow}
\newcommand{\Ot}{\Leftarrow}
\newcommand{\Oto}{\Leftrightarrow}

\renewcommand{\Re}{\operatorname{Re}}
\renewcommand{\Im}{\operatorname{Im}}

\newcommand{\Bild}{\operatorname{Bild}}
\newcommand{\Res}{\operatorname{Res}}

\newcommand{\e}{\text{\rmfamily e}}
\renewcommand{\i}{\text{\rmfamily i}}

\newcommand{\egn}{\varepsilon > 0}
\newcommand{\de}{\delta_\varepsilon}
\newcommand{\Ne}{\N_\varepsilon}

\newcommand{\tdb}{differenzierbar}
\newcommand{\tdbl}{differenzierbar }
\newcommand{\td}[1]{\text{\rmfamily d} #1}

\newcommand{\diff}[3][]{\frac{\text{\rmfamily d}^{#1} #2}{\text{\rmfamily d} #3^{#1}}}
\newcommand{\piff}[3][]{\frac{\partial^{#1} #2}{\partial #3 ^{#1}}}
\newcommand{\fehler}[3][]{\left| \piff[#1]{#2}{#3} \right| * \Delta #3}

%ä, ö, ü in Kapitelüberschriften
\newcommand{\aech}{a\kern-0.17em{e}}
\newcommand{\oech}{o\kern-0.13em{e}}
\newcommand{\uech}{u\kern-0.17em{e}}
%ä, ö, ü in Abschnittsüberschriften
\newcommand{\aese}{a\kern-0.22em{e}}
\newcommand{\oese}{o\kern-0.18em{e}}
\newcommand{\uese}{u\kern-0.22em{e}}

%`Text` im Mathemodus
\mathcode`\`="8000
\begingroup
\uccode`\~`\`
\uppercase{%
\endgroup
\def~#1`{\ \text{#1} \ }}
%Malpunkte
\mathcode`\*="8000
{\catcode`\*\active\gdef*{\cdot}}
%Punkte zu Kommata
\DeclareMathSymbol{.}{\mathord}{letters}{"3B}

\newtheoremstyle{paris}
  {\topsep}% Platzhalter nach oben
  {\topsep}% Platzhalter nach unten
  {\normalfont}% Schriftart im Theoremkörper
  {-25pt}% Seitlicher Einschub
  {\bfseries}% Schriftart in der Theoremüberschrift
  {:\\}% Punkt zwischen Überschrift und Körper
  {5pt plus 1pt minus 1pt}% Platz nach überschrift mit 2pt Ausweichvermögen
  {}% Spezifikationen

\swapnumbers
\theoremstyle{paris}
\newtheorem{df}{Definition}[chapter]
\newtheorem{sz}[df]{Satz}
\newtheorem{genericthm}[df]{\thistheoremname}
\newtheorem{bsp}[df]{Beispiel}
\newenvironment{bw}[1][\textbf{Beweis}]{\begin{proof}[#1]}{\end{proof}}

%Bedienung: \begin{bla}{Satz von Horst}
\newenvironment{bla}[1]
{\renewcommand{\thistheoremname}{#1}\begin{genericthm}}{\end{genericthm}}

%----TEST-Kram----

\def\mathnote#1{%
  \tag*{\rlap{\hspace\marginparsep\smash{\parbox[t]{\marginparwidth}{%
  \footnotesize#1}}}}
}
\def\lmathnote#1{%
  \leqnomode%
  \tag*{\llap{\hspace\marginparsep\smash{\parbox[t]{\marginparwidth}{%
  \footnotesize#1}}}}%
  \reqnomode
}

%----Dokument----

\begin{document}
\begin{titlepage}
\centering
\phantom{Semmelmagus}
\vspace{5cm}
{\tgothfamily\fontsize{60}{72}\selectfont De theoria arcana\par}
\vspace{1cm}
{\Large Paul \textsc{Schwahn}, Jann \textsc{van der Meer}\par}
\vfill
\end{titlepage}
\tableofcontents
\clearpage

\part{Empirice magica}
\part{Viae mathematicae artis magicae}
\chapter[tocentry=Unwahrscheinlichkeitskalkül, head=Unwahrscheinlichkeitskalkül]{Unwahrscheinlichkeitskalk\uech l}
"`[Es ist] einer der gravierenden Fehler derer, die einer theoria magica ablehnend gegenüberstehen, anzunehmen, das Eintreten eines magischen Ereignisses sei im Sinne einer Axiomatisierung der nichtmagischen Naturtheorie unmöglich. [...] Ich aber sage euch, dass [das Auftreten] eines [magischen] Ereignisses lediglich eine Frage der \emph{Unwahrscheinlichkeit}, nicht jedoch der \emph{Unmöglichkeit} ist. Eine [konsistente] theoria magica enthält die [nichtmagische Naturtheorie] als Limes der infinitesimalen\ldots [\textit{Rest unleserlich}]"' (Zit. aus "`\emph{Rohals Apokryphen}"', Universitätsbibliothek zu Punin)

Inwiefern \textsc{Rohal} bereits den Formalismus des Unwahrscheinlichkeitskalküls kannte, bleibt unklar. Er hat jedoch bereits die Begriffe der Unwahrscheinlichkeit und der Unwahrscheinlichkeitsamplitude definiert, also folgen wir hier \textsc{Rohal}.
\begin{bla}{Vorbemerkungen}
Im Folgenden sei stets ein beliebiger Wahrscheinlichkeitsraum $(\Omega,\mathcal{A},P)$ gegeben, d. h. ein Grundraum $\Omega$ aus Elementarereignissen mit einer $\sigma$-Algebra $\mathcal{A}$ über $\Omega$ und einem auf $\mathcal{A}$ definierten Wahrscheinlichkeitsmaß $P$. Als \emph{Ereignis} bezeichnen wir ein Element von $\mathcal{A}$.
\end{bla}
\begin{bla}{Definition: Unwahrscheinlichkeit $I$}
Unter der \emph{Unwahrscheinlichkeit} eines Ereignisses \textbf{E} versteht man den Kehrwert seiner Wahrscheinlichkeit:\\
\begin{equation}
I(\textbf{E}) := \frac{1}{P(\textbf{E})}
\end{equation}
$I$ nennt man \emph{Unwahrscheinlichkeitsmaß}\footnote{Im Grunde genommen ist die Bezeichnung "`Unwahrscheinlichkeitsmaß"' für $I$ irreführend, da sie suggeriert, dass $I$ ein Maß ist, was nicht der Fall ist.} auf $\mathcal{A}$. Somit erhält man den \emph{Unwahrscheinlichkeitsraum} $(\Omega,\mathcal{A},I)$.
\end{bla}
\clearpage
\begin{bla}{Definition: Unwahrscheinlichkeitsamplitude $A$}
Die \emph{Unwahrscheinlichkeitsamplitude} $A$ eines Ereignisses \textbf{E} ist definiert als:
\begin{equation}
A(\textbf{E}) := \ln{I(\textbf{E})}
\end{equation}
\end{bla}
\begin{bla}{Bemerkungen}
Für die Wertebereiche der Größen gelten\footnote{Hierbei stehe \textbf{E} für ein tatsächlich mögliches Ereignis mit $P(\textbf{E}) > 0$, denn wie \textsc{Rohal} thematisieren wir hier die unwahrscheinlichen, nicht die unmöglichen Ereignisse.}:
\begin{itemize}
\item $P(\textbf{E}) \in ]0,1]$
\item $I(\textbf{E}) \in [1,\infty[$
\item $A(\textbf{E}) \in [0,\infty[$
\end{itemize}
Weiterhin gelten zwischen den Größen trivialerweise die folgenden Beziehungen:
\begin{itemize}
\item $I(\textbf{E}) = - \ln P(\textbf{E})$
\item $I(\textbf{E}) = \e ^ {A(\textbf{E})}$
\item $P(\textbf{E}) = I(\textbf{E}) ^{-1} = \e ^{-A(\textbf{E})}$
\end{itemize}
\end{bla}
\begin{bsp}
Der Wahrscheinlichkeitsraum sei $([0,1],\mathcal{B}([0,1]),\lambda)$. Hierbei ist $\mathcal{B}([0,1])$ die \textsc{Borelius}-$\sigma$-Algebra über $[0,1]$ und $\lambda$ das \textsc{Lebexosch}-Maß auf $\mathcal{B}([0,1])$. Dies ist tatsächlich ein Wahrscheinlichkeitsmaß, da es $\sigma$-additiv ist und $\lambda([0,1])=1$. Das zugehörige Unwahrscheinlichkeitsmaß sei $I_\lambda$, sowie die Unwahrscheinlichkeitsamplitude $A_\lambda$. Dann ist beispielsweise für ein beliebiges Intervall $[a,b]\in[0,1]$
\begin{align*}
I_\lambda([a,b])&=\frac{1}{\lambda([a,b])}=\frac{1}{b-a}\\
\intertext{und}
A_\lambda([a,b])&=\ln{I_\lambda([a,b])}=-\ln(b-a)\text{.}
\end{align*}
\end{bsp}
\begin{bla}{Definition: \textsc{Rohal}-Negentropie $R(\textbf{E})$}
Sei \textbf{E} ein Ereignis.
\begin{equation}
R(\textbf{E}) := k_\text{R} * A(\textbf{E})
\end{equation}
heißt \emph{\textsc{Rohal}-Negentropie} von \textbf{E}. Hierbei bezeichne $k_\text{R}$ die \textsc{Rohal}-Konstante.
\end{bla}\clearpage
\part{Metaphysica theoretica}
\part{Implicationes arcanobiologicae}

%\input{}\clearpage

\end{document}
